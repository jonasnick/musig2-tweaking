\newif\iffull % false is IACR submission
\newif\iffull % false is IACR submission
\documentclass[a4paper,orivec,oribibl,english]{llncs}
\usepackage[top = 3 cm , bottom = 3 cm , left = 3 cm , right = 3 cm]{geometry}

%----PACKAGES-------------------

% \usepackage[notref,notcite,color]{showkeys}

\usepackage{etex}
\usepackage[utf8]{inputenc}
\usepackage[T1]{fontenc}
\usepackage{lmodern}
\usepackage[english]{babel}
\usepackage{fancyvrb}
\usepackage{fvextra}
\usepackage{csquotes}

\usepackage{todonotes}

\usepackage{xspace}
\usepackage{mathdots}
\usepackage{array}
\usepackage{subeqnarray}
\usepackage{tabularx}
\usepackage{multirow,makecell}
\usepackage{amsmath}
\usepackage{amssymb}
\let\proof\relax % trick to combine amsthm with llncs - amsthm is useful for \qedhere
\let\endproof\relax
\usepackage{amsthm} % don't load it when using LNCS style, unless using the relax trick
\usepackage{breqn} % load breqn this after ams packages, https://tex.stackexchange.com/a/325264
\usepackage{bm}

% LaTeX positions the subscripts slightly deeper if a superscript is present
% (and ' is just short for ^{\prime}, so it's a superscript).
% This makes equations like h_{agg} = h'_{agg} look strange, and we have lots
% of them and similar math in the ROM proof.
% The normal fix is to write h^{}_{agg} = h'_{agg} but that doesn't work for our
% \hagg etc. macros because it could lead to double superscripts.
% The subdepth package simply makes all subscript the same depth.
% That's somewhat inelegant because it's a global change but it looks good in practice.
\usepackage[low-sup]{subdepth}

\iffull % for changelog
\usepackage[iso]{isodate}
\fi

\usepackage{placeins}
% \usepackage{enumerate}  % load enumerate OR enumitem
\usepackage{enumitem}  % allows to stop and resume a list
% \usepackage{listings} % use listings OR tcolorbox with listings library
\usepackage{tcolorbox}
\tcbuselibrary{listings}
\usepackage{microtype}
\usepackage{url}
\urlstyle{same}
\usepackage[normalem]{ulem}
% \usepackage{algorithm}
% \usepackage{algpseudocode}
\usepackage{thm-restate}

\usepackage{changepage}                 % adjust margins for selected portions
% wide page for side by side figures, tables, etc
\newlength{\offsetpage}
\iffull
\setlength{\offsetpage}{0cm}
\else
\setlength{\offsetpage}{2.0cm}
\fi
\newenvironment{widepage}{\begin{adjustwidth}{-\offsetpage}{-\offsetpage}%
    \addtolength{\textwidth}{2\offsetpage}}%
  {\end{adjustwidth}}

\usepackage{tikz}
\usepgflibrary[arrows]
\usetikzlibrary{arrows,decorations.pathreplacing}
\usetikzlibrary{calc}
\usetikzlibrary{positioning}
\usetikzlibrary{shapes.geometric}

% \usepackage[pdftex]{color,graphicx,rotating}  % don't load it if tikz is loaded 
\graphicspath{{figs/}}
\DeclareGraphicsExtensions{.jpg,.png,.pdf}
\pdfcompresslevel=9


% Fix LNCS template to produce bookmarks for single paper
% https://tex.stackexchange.com/a/47410/61094
\usepackage{etoolbox}
\makeatletter
\let\llncs@addcontentsline\addcontentsline
\patchcmd{\maketitle}{\addcontentsline}{\llncs@addcontentsline}{}{}
\patchcmd{\maketitle}{\addcontentsline}{\llncs@addcontentsline}{}{}
\patchcmd{\maketitle}{\addcontentsline}{\llncs@addcontentsline}{}{}
\setcounter{tocdepth}{2}
\makeatother
\usepackage{hyperref}
\usepackage{bookmark}

\hypersetup{
  colorlinks=true,
  citecolor=blue,
  linkcolor=blue,
  urlcolor=blue,
  bookmarksopen,
  bookmarksdepth=3,
  pdftitle={MuSig2: Simple Two-Round Schnorr Multi-Signatures}
}


\usepackage[
  backend=bibtex,
  style=alphabetic,
  minalphanames=3,
  maxalphanames=3,
  giveninits=true,
  maxbibnames=99,
  date=year
]{biblatex}


\usepackage[lambda,
            advantage,
            adversary,
            probability,
            operators,
            keys,
            asymptotics,
            sets]{cryptocode}
\renewcommand{\pckeystyle}[1]{{\ensuremath{\mathit{\protect\vphantom{p}#1}}}}

\usepackage{nccfoots}

\newcommand{\tnote}[1]{\todo[color=yellow!30, inline]{\textbf{Tim's note:} #1}}
\newcommand{\pnote}[1]{\todo[color=green!30, inline]{\textbf{Pieter's note:} #1}}
\newcommand{\jnote}[1]{\todo[color=orange!30, inline]{\textbf{Jonas' note:} #1}}
\newcommand{\dnote}[1]{\todo[color=purple!30, inline]{\textbf{Duc's note:} #1}}

\renewcommand{\pcadvantagesuperstyle}[1]{\operatorname{{#1}}}

% a lower version of \widetilde that does not interfere with the lineheight
% when applied to \mathsf{pk}.
% from https://tex.stackexchange.com/a/3895
\makeatletter
\newcommand*\wt[2][0.2ex]{%
        \begingroup
        \mathchoice{\wt@helper{#1}{#2}{\displaystyle}{\textfont}}
                   {\wt@helper{#1}{#2}{\textstyle}{\textfont}}
                   {\wt@helper{#1}{#2}{\scriptstyle}{\scriptfont}}
                   {\wt@helper{#1}{#2}{\scriptscriptstyle}{\scriptscriptfont}}%
        \endgroup
        #2%
}
\newcommand*\wt@helper[4]{%
        \def\currentfont{\the#41}%
        \def\currentskewchar{\char\the\skewchar\currentfont}%
        \setbox\tw@\hbox{\currentfont#2\currentskewchar}%
        \dimen@ii\wd\tw@
        \setbox\tw@\hbox{\currentfont#2{}\currentskewchar}%
        \advance\dimen@ii-\wd\tw@
        \rlap{\raisebox{-#1}{$\m@th#3\kern\dimen@ii\widetilde{\phantom{#2}}$}}%
}
\makeatother

%----GENERAL COMMANDS----------------

\newcommand{\defin}{\mathrel{\mathop=^{\rm def}}}
% a nicer := sign ("is defined as"), http://tex.stackexchange.com/questions/4216/how-to-typeset-correctly
\makeatletter
\DeclareRobustCommand{\defeq}{\mathrel{\rlap{%
  \raisebox{0.3ex}{$\m@th\cdot$}}%
  \raisebox{-0.3ex}{$\m@th\cdot$}}%
  =}
\DeclareRobustCommand{\eqdef}{=\mathrel{\rlap{%
  \raisebox{0.3ex}{$\m@th\cdot$}}%
  \raisebox{-0.3ex}{$\m@th\cdot$}}%
  }
\makeatother\newcommand{\bool}{\{0,1\}}
\newcommand{\str}{\bool^*}
% \newcommand{\poly}{{\tt poly}}
% \newcommand{\negl}{{\tt negl}}
\newcommand{\nat}{\mathbb{N}}
\newcommand{\integ}{\mathbb{Z}}
\newcommand{\real}{\mathbb{R}}
\newcommand{\complex}{\mathbb{C}}
\newcommand{\field}{\mathbb{F}}
\newcommand{\ve}{\varepsilon}
\newcommand{\la}{\leftarrow}
\newcommand{\pr}[1]{\Pr\left[ #1 \right]}
\newcommand{\cond}{\middle|}
\newcommand{\ignore}[1]{}
\iffull
\newcommand{\hd}[1]{\paragraph{\sc #1}}
\else
\newcommand{\hd}[1]{\paragraph{#1}} % don't use personalized style in anonymous submission
\fi

\newcommand{\vect}[1]{\vec{#1}}

%----SPECIFIC COMMANDS-----------------

\newcommand{\mathsc}[1]{{\normalfont\textsc{#1}}}
\newcommand{\musig}{\ensuremath{\mathsf{MuSig}}\xspace}
\newcommand{\musigtwo}{\ensuremath{\mathsf{MuSig2}}\xspace}
\newcommand{\musigtwotweak}{\ensuremath{\mathsf{MuSig2Tweak}}\xspace}
\newcommand{\musigtwontweak}{\ensuremath{\mathsf{MuSig2NaiveTweak}}\xspace}
\newcommand{\musigtwostar}{\ensuremath{\mathsf{MuSig2^*}}\xspace}
\newcommand{\musigbroken}{\ensuremath{\mathsf{InsecureMuSig}}\xspace}
\newcommand{\musigdn}{\ensuremath{\mathsf{MuSig\textnormal{-}DN}}\xspace}
\newcommand{\mbcj}{\ensuremath{\mathsf{mBCJ}}\xspace}
\newcommand{\bn}{\ensuremath{\mathsf{BN}}\xspace}
\newcommand{\cosi}{\ensuremath{\mathsf{CoSi}}\xspace}
\newcommand{\msdlpop}{\ensuremath{\mathsf{MSDL\textnormal{-}pop}}\xspace}
\newcommand{\frost}{\ensuremath{\mathsf{FROST}}\xspace}
\newcommand{\frosti}{\ensuremath{\mathsf{FROST\textnormal{-}Interactive}}\xspace}
\newcommand{\dwms}{\ensuremath{\mathsf{DWMS}}\xspace}
\newcommand{\eufcma}{\ensuremath{\mathsf{EUF\textnormal{-}CMA}}\xspace}
\newcommand{\eufcmaalt}{\ensuremath{\mathsf{EUF\textnormal{-}CMA'}}\xspace}
\newcommand{\correct}{\ensuremath{\mathsf{CORRECT}}\xspace}
\newcommand{\dl}{\ensuremath{\mathsf{DL}}\xspace}
\newcommand{\omdl}{\ensuremath{\mathsf{OMDL}}\xspace}
\newcommand{\aomdl}{\ensuremath{\mathsf{AOMDL}}\xspace}
\newcommand{\grgen}{\mathsf{GrGen}}
\newcommand{\setup}{\mathsf{Setup}}
\newcommand{\keygen}{\mathsf{KeyGen}}
\newcommand{\keytweak}{\mathsf{KeyTweak}}
\newcommand{\sign}{\mathsf{Sign}}
\newcommand{\signagg}{\mathsf{SignAgg}}
\newcommand{\ver}{\mathsf{Ver}}
\newcommand{\musigcoef}{\mathsf{KeyAggCoef}}
\newcommand{\keyagg}{\mathsf{KeyAgg}}
\newcommand{\keyaggbroken}{\mathsf{InsecureKeyAgg}}
\newcommand{\inpgen}{\mathsf{InpGen}}
% \newcommand{\pk}{\mathsf{pk}}
% \newcommand{\sk}{\mathsf{sk}}
% \newcommand{\GG}{\mathbb{G}}
\newcommand{\apk}{\wt{\pk}}
\newcommand{\anL}{\langle L \rangle}
\newcommand{\anS}{\langle S \rangle}
\newcommand{\Hc}{H_{\mathrm{c}}}
\newcommand{\ctrh}{\mathit{ctrh}}
\newcommand{\ctrs}{\mathit{ctrs}}
\newcommand{\ctrhagg}{\mathit{ctrh}_{\mathrm{agg}}}
\newcommand{\ctrhsig}{\mathit{ctrh}_{\mathrm{sig}}}
\newcommand{\ctrhnon}{\mathit{ctrh}_{\mathrm{non}}}
\newcommand{\ctr}{\mathit{ctr}}
\newcommand{\tX}{\widetilde{X}}
\newcommand{\tx}{\tilde{x}}
\newcommand{\ba}{\vect{a}}
\newcommand{\bh}{\vect{h}}
\newcommand{\bv}{\vect{v}}
\newcommand{\DL}{\mathsc{DLog}}
\newcommand{\ADL}{\mathsc{DLog}}
\newcommand{\fork}{\mathsf{Fork}}
\newcommand{\inp}{\mathit{inp}}
\newcommand{\out}{\mathit{out}}
\newcommand{\acc}{\mathit{acc}}
\newcommand{\frk}{\mathit{frk}}
\newcommand{\bad}{\mathsf{Bad}}
\newcommand{\alert}{\mathsf{Alert}}
\newcommand{\tr}{\mathtt{true}}
\newcommand{\false}{\mathtt{false}}
\newcommand{\ev}{\mathbf{E}}
\newcommand{\Hagg}{\mathsf{H}_{\mathrm{agg}}}
\newcommand{\Hsig}{\mathsf{H}_{\mathrm{sig}}}
\newcommand{\Hnon}{\mathsf{H}_{\mathrm{non}}}
\newcommand{\Ht}{\mathsf{H}_{\mathrm{tweak}}}
\newcommand{\HHagg}{\overline{\mathsf{H}}_{\mathrm{agg}}}
\newcommand{\HHsig}{\overline{\mathsf{H}}_{\mathrm{sig}}}
\newcommand{\HHnon}{\overline{\mathsf{H}}_{\mathrm{non}}}
\newcommand{\Tagg}{T_{\mathrm{agg}}}
\newcommand{\Tsig}{T_{\mathrm{sig}}}
\newcommand{\Tnon}{T_{\mathrm{non}}}
\newcommand{\Takey}{T_{\mathrm{akey}}}
\newcommand{\Tsigrep}{T_{\mathrm{sigrep}}}
\newcommand{\Ttweaks}{T_{\mathrm{tweaks}}}
\newcommand{\Tagginv}{T_{\mathrm{agginv}}}
\newcommand{\Tqueried}{T_{\mathrm{queried}}}
\newcommand{\hagg}[1]{h_{\mathrm{agg},#1}}
\newcommand{\haggSingle}{h_{\mathrm{agg}}}
\newcommand{\hnon}[1]{h_{\mathrm{non},#1}}
\newcommand{\hsig}[1]{h_{\mathrm{sig},#1}}
\newcommand{\hsigSingle}{h_{\mathrm{sig}}}
\newcommand{\iagg}{i_{\mathrm{agg}}}
\newcommand{\jagg}{j_{\mathrm{agg}}}
\newcommand{\isig}{i_{\mathrm{sig}}}
\newcommand{\jsig}{j_{\mathrm{sig}}}
\newcommand{\sessions}{S}
\newcommand{\msg}{\mathit{out}}
\renewcommand{\state}{\mathit{state}}
\newcommand{\queries}{\mathsf{Q}}
\newcommand{\js}{\mathsf{J}}
\ifdefined\pcgame % HACK to make this compile for both Yannick as well as others
\renewcommandx{\pcgame}[4][3=\adv,4=(\secpar)]{{\operatorname{#1}_{#2}^{#3}#4}}
\else
\newcommandx{\pcgame}[4][3=\adv,4=(\secpar)]{{\operatorname{#1}_{#2}^{#3}#4}}
\fi
\unless\ifdefined\pclinecomment
\newcommand{\pclinecomment}[2][0em]{\hspace{#1}{\mbox{/\!\!/ } \text{\scriptsize#2}}}
\fi
\newcommand{\emptylist}{(\,)}
\newcommand{\signo}{\mathsc{Sign}}
\newcommand{\chalo}{\mathsc{Ch}}
\newcommand{\keyaggo}{\mathsc{KeyAgg}}
\newcommand{\pcsc}{\,;~}
\newcommand{\agmacc}{26q^3/2^{\secpar}}
\newcommand{\agmtime}{t + O(qN) \cdot t_{\rm exp} + O(q^3)}
\newcommand{\startime}{t + O(qN) t_{\rm exp}}
\newcommand{\q}{2q_h+q_s+1}
\newcommand{\param}{\mathit{par}} % public parameters of the multisig scheme
\newcommand{\gparam}{(\GG,p,g)}
\newcommand{\ssum}{\textstyle \sum}
\newcommand{\sprod}{\textstyle \prod}
\newcommand{\Zpp}{\mathbb{Z}_p} % currently Z_p, could be redefined to be a subset of Z_p (if we generalize the range of Hsig, Hagg to a subset of Z_p)
% HACK What was \pp before?
\renewcommand{\pp}{p} % currently p, could be redefined to the size of \Zpp
\newcommand{\game}{\mathsf{Game}}
\newcommand{\tg}[3]{\tilde{\gamma}_{#1,#2,#3}}
% session index and session max index
\newcommand{\si}{k}
\newcommand{\smax}{k_{\mathrm{max}}}
\newcommand{\lmax}{\ell_{\mathrm{max}}}
\newcommand{\tK}{\widetilde{K}}
\newcommand{\before}{\rightsquigarrow}

% Bad events
\newcommand{\badcom}{\mathsf{BadCom}}
\newcommand{\badprog}{\mathsf{BadProg}}
\newcommand{\badorder}{\mathsf{BadOrder}}
\newcommand{\keycoll}{\mathsf{KeyColl}}
\newcommand{\lindep}{\mathsf{LinDep}}
\newcommand{\aggfail}{\mathsf{AggFail}}
\newcommand{\noncoll}{\mathsf{NonColl}}
\newcommand{\badtsig}{\mathsf{BadTsig}}
\newcommand{\badtnon}{\mathsf{BadTnon}}
\newcommand{\checkagg}{\mathsf{CheckTagg}}
\newcommand{\checknon}{\mathsf{CheckTnon}}
\newcommand{\checksig}{\mathsf{CheckTsig}}
\newcommand{\issecond}{\mathsf{IsSecond}}
\newcommand{\transl}{\mathsf{transl}}


\newcommand{\kmax}{k_{\mathrm{max}}}
  
\newcommand{\pcdoublebox}[1]{%
  {\setlength{\fboxsep}{1pt}\fbox{%
  {\setlength{\fboxsep}{3pt}\fbox{%
  $\displaystyle#1$%
  }}%
  }}%
}%

\newcommand{\pcboxblock}[1]{%
\hspace{-2pt}{\setlength{\fboxsep}{2pt}\fbox{\pseudocode{#1}}}%
}

\newcommand{\pcdoubleboxblock}[1]{%
\hspace{-3pt}{\setlength{\fboxsep}{1pt}\fbox{\setlength{\fboxsep}{2pt}\fbox{\pseudocode{#1}}}}%
}

\newcommand{\textmod}[1]{{\textcolor{red}{\textbf{#1}}}}
\newcommand{\mathmod}[1]{{\textcolor{red}{\mathbf{#1}}}}
\newcommand{\stkout}[1]{\ifmmode\text{\sout{\ensuremath{#1}}}\else\sout{#1}\fi}


\begin{document}
\title{
    Breaking MuSig2 with tweaking
}
\maketitle

\section{Modified Scheme}
See Figure \ref{fig:ms}.
\begin{figure}
 \begin{tcolorbox}[boxsep=-1mm]
  \begin{pchstack}[center]
   \begin{pcvstack}
    \procedure{$\setup(\secparam)$}{%
      \gparam \gets \grgen(\secparam) \\
      \text{Select three hash functions}\\
      \t \Hagg, \Hnon, \Hsig:\str\to\integ_p\\
      \param \defeq (\gparam, \Hagg, \Hnon, \Hsig)\\
      \pcreturn \param
    }
    \pcvspace
    \procedure{$\keygen()$}{%
      % (p,\GG,g) \defeq \param \\
      x \sample \ZZ_p \pcsc X \defeq g^x \\
      \sk \defeq x \pcsc \pk \defeq X \\
      \pcreturn (\sk,\pk)
    }
    \pcvspace
    \procedure{$\musigcoef(L,X_i)$}{%
      \pcreturn \Hagg(L,X_i)
    }
    \pcvspace
    \procedure{$\keyagg(L)$}{%
      \{X_1,\ldots,X_n\} \defeq L \\
      \pcfor i \defeq 1 \ldots n \pcdo \\
      \t a_i \defeq \musigcoef(L,X_i)\\
      \pcreturn \tX \defeq \sprod_{i=1}^n X_i^{a_i}
    }
    \pcvspace
    \procedure{$\ver(\apk,m,\sigma)$}{%
      % (p,\GG,g) \defeq \param \\
      \tX \defeq \apk \pcsc (R,s) \defeq \sigma \\
      c \defeq \Hsig(\tX,R,m) \\
      \pcreturn (g^s = R \tX^c)
    }
    \pcvspace
    \procedure{$\sign()$}{%
%       (p,\GG,g) \defeq \param \\
      \pclinecomment{Local signer has index $1$.} \\
      \pcfor j \defeq 1 \ldots \nu \pcdo \\
        \t r_{1,j} \sample \ZZ_p \pcsc R_{1,j} \defeq g^{r_{1,j}} \\
      \msg_1 \defeq (R_{1,1},\ldots,R_{1,\nu}) \\
      \state_1 \defeq (r_{1,1},\ldots,r_{1,\nu}) \\
      \pcreturn (\msg_1,\state_1)
    }
   \end{pcvstack}
   \iffull \pchspace[5em] \else \pchspace[0.7em] \fi
   \begin{pcvstack}
   \procedure{$\signagg(\msg_1,\ldots,\msg_n)$}{%
     \pcfor i \defeq 1 \ldots n \pcdo \\
     \t (R_{i,1}, \ldots, R_{i,\nu}) \defeq \msg_i\\
     \pcfor j \defeq 1 \ldots \nu \pcdo \\
     \t R_j \defeq \sprod_{i=1}^n R_{i,j}\\
     \pcreturn \msg \defeq (R_1, \ldots, R_\nu)
    }
    \pcvspace
    \procedure{$\sign'(\state_1,\msg,\sk_1,m,(\pk_2,\ldots,\pk_n), \textcolor{red}{C})$}{%
      \pclinecomment{$\sign'$ must be called at most once per $\state_1$.} \\
      (r_{1,1},\ldots,r_{1,\nu}) \defeq \state_1 \\
      x_1 \defeq \sk_1 \pcsc X_1' \defeq g^{x_1} \pcsc \textcolor{red}{t \defeq \Hcon(X_1', C) \pcsc X_1 \defeq X_1' g^t} \\
      (X_2, \ldots, X_n) \defeq (\pk_2, \ldots, \pk_n)\\
      L \defeq \{X_1,\ldots,X_n\} \\
      a_1 \defeq \musigcoef(L,X_1)\\
      \tX \defeq \keyagg(L)\\
      (R_1, \ldots, R_\nu) \defeq \msg\\
      b \defeq \Hnon(\tX,(R_1,\ldots,R_{\nu}),m) \\
      R \defeq \sprod_{j=1}^{\nu} R_j^{b^{j-1}} \\
      c \defeq \Hsig(\tX,R,m) \\
      s_1 \defeq c a_1 \textcolor{red}{(x_1 + t)} + \ssum_{i=1}^{\nu} r_{1,j} b^{j-1} \bmod p \\
      \state'_1 \defeq R \pcsc \msg'_1 \defeq s_1 \\
      \pcreturn (\state'_1,\msg'_1)
    }
    \pcvspace
    \procedure{$\signagg'(\msg'_1,\ldots,\msg'_n)$}{%
      (s_1, \ldots, s_n) \defeq (\msg'_1, \ldots, \msg'_n)\\
      s \defeq \ssum_{i=1}^n s_i \bmod p \\
      \pcreturn \msg' \defeq s
    }
    \pcvspace
    \procedure{$\sign''(\state'_1,\msg')$}{%
      R \defeq \state'_1 \pcsc s \defeq \msg'\\
      \pcreturn \sigma \defeq (R,s)
    }
   \end{pcvstack}
  \end{pchstack}
 \end{tcolorbox}
 \caption{The modified multi-signature scheme $\musigtwotweak[\grgen,\nu]$. Modifications in \textcolor{red}{red}.}
 \label{fig:ms}
\end{figure}

\section{Wagner}
The attack relies on Wagner's algorithm for solving the Generalized Birthday Problem, which can be defined as follows for the purpose of this paper:
Given a constant value $t\in\integ_p$, an integer $\smax$,
and access to random oracle $H$ mapping onto $\integ_p$,
find a set $\{q_1, \ldots, q_{\smax}\}$ of $\smax$ queries  such that
\(
  \sum_{\si=1}^{\smax} H(q_{\si}) = t.
\)
For $\smax = 2^{\sqrt{\log_2(p)}-1}$ the complexity of this algorithm is $O(2^{2\sqrt{\log_2(p)}})$.
\jnote{Perhaps can use BLOR? ``If the attacker is able to open more sessions concurrently, the improved polynomial-time attack by Benhamouda \emph{et al.}~\cite{add:BLOR20} assumes $\smax > \log_2 p$ sessions, but then has complexity $O(\smax \log_2 p)$ and a negligible running time in practice.''}

\section{Attack}
The attack proceeds as follows.
The adversary opens $\smax = 2^{\sqrt{\log_2(p)}-1}$ concurrent signing sessions and receives $\smax$ nonce pairs $R_{1,1}^{(1)}, R_{1,2}^{(1)}, \ldots, R_{1,1}^{(\smax)}, R_{1,2}^{(\smax)}$ from the honest signer with public key $X_1=g^{x_1}$.

The adversary draws $\lmax \in O(2^{2\sqrt{\log_2(p)}})$ values $C_1, \ldots, C_\lmax$ at random.
Let  $L = \{X_1g^{t^{(1)}}, \ldots, X_1g^{t^{(\lmax)}}\}$ where $t^{(\ell)}$ are the tweaks and  $\tX=KeyAgg(L)$ be the corresponding aggregate public key.
\jnote{The attack probably doesn't require such large $L$ but can work with multiple smaller multisets of keys}

Given $R_1 = \sum_{\si=1}^\smax R_{1,1}^{(\si)}, R_2 = \sum_{\si=1}^\smax R_{1,2}^{(\si)}$ and a forgery target message $m^*$, the adversary computes $b = \Hnon(\tX,(R_1,R_2),m)$ and $R^*  =  \prod_{\si=1}^{\smax} R_{1,1}^{(\si)} (R_{1,2}^{(\si)})^b$.
and uses Wagner's algorithm to find a $t^{(\ell)}, 1 \le \ell \le \lmax$ for each session $\si, 1\le \si \le \smax$ (in other words a function $f:[1, \smax] \rightarrow [1,\lmax]$) such that

\begin{equation}\label{eq:wagner-musig}
  \sum_{\si=1}^{\smax} \underbrace{\Hagg(L, X_1g^{t^{(f(\si))}})}_{=:\,a_1^{(\si)}} \underbrace{\Hsig(\tX, R^*, m)}_{=:\,c^{(\si)}}
  = \underbrace{\Hsig(X_1, R^*, m^*)}_{=:\,c^*} .
\end{equation}
The honest signer will reply with $\kmax$ partial signatures $s_1^{(\si)} = r_{1,1}^{(\si)} + b r_{1,2}^{(\si)} +  c^{(\si)} \cdot a_1^{(\si)} (x_1 + t^{(f(\si))})$.
The adversary is able to obtain
\begin{align}\label{eq:s-sum}
  s_1^{*'} &=  \sum_{\si=1}^\smax s_1^{(\si)}\\
  &= \sum_{\si=1}^\smax \left( r_{1,1}^{(\si)} br_{1,2}^{(\si)}\right) +  \left(\sum_{\si=1}^\smax c^{(\si)} a_1^{(\si)}\right) \cdot x_1 + \sum_{\si=1}^\smax c^{(\si)} a_1^{(\si)}t^{(f(\si))} \\
  &= \log_g(R^*) +  c^* x_1 + \sum_{\si=1}^\smax c^{(\si)} a_1^{(\si)}t^{(f(\si))}
\end{align}
where the last equality follows from Equation~\eqref{eq:wagner-musig}.
Then the adversary can subtract the unwanted term as follows
\[
  s_1^* =  s_1^{*'} - \sum_{\si=1}^\smax c^{(\si)} a_1^{(\si)}t^{(f(\si))}
\]
to obtain $(R^*, s^*)$, a valid forgery on message $m^*$ for public key $X_1$.

\section{(Preliminary) Conclusion}
However, Equation~\eqref{eq:s-sum} seems to indicate that if the honest signer signs with multiple random keys instead of tweaked keys then this attack does not apply.
The adversary would only be able to obtain a $s_1^* = \log_g(R^*) +  \sum_{\si=1}^\smax c^{(\si)} a_1^{(\si)} x^{(f(\si))}$.
\end{document}